\documentclass[12pt]{article}

\usepackage[english, russian]{babel} 
\usepackage{fontspec}
\setmainfont[Mapping=tex-text-ms]{PonomarUnicode}

\begin{document}
	\begin{center}
		Стихотвореніе о ятѣ 
	\end{center}
	\vfill
	\begin{flushleft}
		\begin{verse}
			Бѣлый, блѣдный, бѣдный бѣсъ \\
			Убѣжалъ голодный въ лѣсъ. \\
			Лѣшимъ по лѣсу онъ бѣгалъ, \\
			Рѣдькой съ хрѣномъ пообѣдалъ \\
			И за горькій тотъ обѣдъ \\
			Далъ обѣтъ надѣлать бѣдъ. \\
			Вѣдай, братъ, что клѣть и клѣтка, \\
			Рѣшето, рѣшетка, сѣтка, \\
			Вѣжа и желѣзо съ ять, -- \\
			Такъ и надобно писать. \\
			Наши вѣки и рѣсницы \\
			Защищаютъ глазъ зѣницы, \\
			Вѣки жмуритъ цѣлый вѣкъ \\
			Ночью каждый человѣкъ... \\
			Вѣтеръ вѣтки поломалъ, \\
			Нѣмецъ вѣники связалъ, \\
			Свѣсилъ вѣрно при промѣнѣ, \\
			За двѣ гривны продалъ въ Вѣнѣ. \\
			Днѣпръ и Днѣстръ, какъ всѣмъ извѣстно, \\
			Двѣ рѣки въ сосѣдствѣ тѣсномъ, \\
			Дѣлитъ области ихъ Бугъ, \\
			Рѣжетъ съ сѣвера на югъ. \\
			Кто тамъ гнѣвно свирѣпѣетъ? \\
			Крѣпко сѣтовать такъ смѣетъ? \\
			Надо мирно споръ рѣшить \\
			И другъ друга убѣдить... \\
			Птичьи гнѣзда грѣхъ зорить, \\
			Грѣхъ напрасно хлѣбъ сорить, \\
			Надъ калѣкой грѣхъ смѣяться, \\
			Надъ увѣчнымъ издѣваться...
		\end{verse}
	\end{flushleft}
	\vfill
	\begin{flushleft}
		Проф. Н. К. Кульманъ. Методика русскаго языка. — 3-е изд. — СПб.:~изданіе Я.~Башмакова~и~Ко, 1914. — С. 182.
	\end{flushleft}
\end{document}